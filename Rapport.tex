\documentclass[11pt]{article}

\usepackage[utf8]{inputenc} % Required for inputting international characters
\usepackage[T1]{fontenc} % Output font encoding for international characters
\usepackage[francais]{babel}

\usepackage{mathpazo} % Palatino font

\begin{document}

%----------------------------------------------------------------------------------------
%	TITLE PAGE
%----------------------------------------------------------------------------------------

\begin{titlepage} % Suppresses displaying the page number on the title page and the subsequent page counts as page 1
	\newcommand{\HRule}{\rule{\linewidth}{0.5mm}} % Defines a new command for horizontal lines, change thickness here
	
	\center % Centre everything on the page
	
	%------------------------------------------------
	%	Headings
	%------------------------------------------------
	
	\textsc{\LARGE Mines Nancy}\\[1.5cm] % Main heading such as the name of your university/college
	
	\textsc{\Large Formation 2A FICM}\\[0.5cm] % Major heading such as course name
	
	\textsc{\large }\\[0.5cm] % Minor heading such as course title
	
	%------------------------------------------------
	%	Title
	%------------------------------------------------
	
	\HRule\\[0.4cm]
	
	{\huge\bfseries Rapport de projet 2A : Extraction d'axe médian pour l'impression 3D}\\[0.4cm] % Title of your document
	
	\HRule\\[1.5cm]
	
	%------------------------------------------------
	%	Author(s)
	%------------------------------------------------
	
	\begin{minipage}{0.4\textwidth}
		\begin{flushleft}
			\large
			\textit{Élèves}\\
			Serosh \textsc{Deljam} \\
			Aymeric \textsc{Bouzigues}
		\end{flushleft}
	\end{minipage}
	~
	\begin{minipage}{0.4\textwidth}
		\begin{flushright}
			\large
			\textit{Encadrant}\\
			Cédric \textsc{Zanni} % Supervisor's name
		\end{flushright}
	\end{minipage}
	
	% If you don't want a supervisor, uncomment the two lines below and comment the code above
	%{\large\textit{Author}}\\
	%John \textsc{Smith} % Your name
	
	%------------------------------------------------
	%	Date
	%------------------------------------------------
	
	\vfill\vfill\vfill % Position the date 3/4 down the remaining page
	
	{\large Soutenance du 13 juin 2018} % Date, change the \today to a set date if you want to be precise
	
	%------------------------------------------------
	%	Logo
	%------------------------------------------------
	
	%\vfill\vfill
	%\includegraphics[width=0.2\textwidth]{placeholder.jpg}\\[1cm] % Include a department/university logo - this will require the graphicx package
	 
	%----------------------------------------------------------------------------------------
	
	\vfill % Push the date up 1/4 of the remaining page
	
\end{titlepage}

\section{Introduction et motivation du sujet de ce projet}

L'impression 3D est un procédé de plus en plus populaire car il permet de créer {\it n'importe quelle pièce} en quelques heures, là où il était auparavant d'avoir nécessaire plusieurs machines faisant chacune sa fonction bien spécifique. Pour ce procédé, un filament de matière\footnote{Bien souvent du plastique, pour des raison de coût.} est chauffé afin d'être rendu malléable et est appliqué sur le modèle que l'on souhaite imprimé. Le matériau se refroidira par la suite, pour redevenir solide et former le modèle souhaité. 

\end{document}
\grid